\chapter{Сбор данных}
Для решения задачи распознавания возраста человека по фотографии лица был проведён анализ доступных открытых наборов данных (датасетов), содержащих изображения лиц с аннотированным возрастом.  


\section{Публичные датасеты}
В рамках работы используются и рассматриваются следующие открытые датасеты:
\begin{itemize}
	\item UTKFace;
	\item APPA-Real Age;
\end{itemize}
\subsection*{UTKFace}
UTKFace - один из наиболее распространённых наборов данных для задач оценки возраста и пола.  
Он содержит более 20\,000 изображений лиц людей в возрасте от 0 до 116 лет, снятых в различных условиях освещения, ракурсах и с разным фоном.  \cite{zhifei2017cvpr}
Каждое изображение имеет аннотацию, закодированную непосредственно в названии файла в формате:
\begin{center}
	\smallcode{[age]\_[gender]\_[race]\_[date\_time].jpg}
\end{center}
\begin{itemize}
	\item \smallcode{age} — возраст человека (целое число);
	\item \smallcode{gender} — пол (0 — мужчина, 1 — женщина);
	\item \smallcode{race} — этническая принадлежность (0–4, пять категорий);
	\item \smallcode{date\_time} — метка времени.
\end{itemize}

Изображения имеют размер: 200 на 200 пикселей.

\subsection*{APPA-Real Age}
APPA-Real Age - содержит 7591 изображение с указанием реального и предполагаемого возраста. Общее количество предполагаемых голосов составляет около 250 000. В среднем на каждое изображение приходится около 38 голосов, что делает средний предполагаемый возраст очень стабильным (0,3 стандартной ошибки от среднего значения). \cite{agustsson2017appareal}

Каждое изображение имеет порядковый номер, а реальный возраст указан в отдельном \smallcode{csv} файле: 
\begin{itemize}
	\item \smallcode{image name} — порядковый номер изображения;
	\item \smallcode{real\_age} — реальный возраст;
	\item \smallcode{apparent\_age} — воспринимаемый возраст;
\end{itemize}


Изображения имеют различный размер.\\
В данной работе будем использовать только метки реального возраста.

Распределение возрастов людей на фотографиях приведено на \hyperref[fig:age-dist]{Рисунке \ref*{fig:age-dist}}.
\begin{figure}[h!]
	\centering
	\includegraphics[width=1\linewidth]{../artifacts/age_histogram.pdf}
	\caption{Распределения возраста лиц на фотографиях из датасетов APPA, UTKFace}
	\label{fig:age-dist}
\end{figure}


\chapter{Предобработка данных}
Для выбранных датасетов будут выполнены следующие шаги для приведения данных к единому виду:
\begin{enumerate}
	\item Загрузка данных из открытых источников.
	\item Конкатенация датасетов с консистентным именем файла: \smallcode{image\_[number]\_[real\_age].jpg}
	\item Создание \smallcode{csv}-файла с метками для изображений.
	\item Приведение изображений к единому размеру.
	\item Разбиение на тестовую, валидационную и тестовые подвыборки.
\end{enumerate}

\subsection*{Загрузка данных из открытых источников}
Для загрузки датасетов был реализовать скрипт на языке \smallcode{bash}. Исходный код скрипта приведен в \hyperref[lst:datasetinit]{Приложении А}.


При успешном выполнении скрипта датасет из изображений представлен в предварительном формате:
\begin{lstlisting}[caption={Иерархия файлов проекта}, label={lst:datasetinit},columns=fullflexible, keepspaces=true]
...
datasets/
	UTKFace/		# Изображения (метки внутри названия файла)
	appa/
		*.csv 		# .csv файлы с метками
		test/		# Изображения связаны с метко й через название файла
		train/
		val/
\end{lstlisting}

\subsection*{Конкатенация датасетов}
Для конкатенации датасетов реализован скрипт на языке \smallcode{Python} \cite{cnn-age-recognize-concat-datasets}.


Скрипт собирает изображения из скачанных датасетов в единую директорию. Имена файлов имеют консистентные названия:\\
\smallcode{image\_[number]\_[real\_age].jpg}.

\subsection*{Создание файла с метками}
Для создания \smallcode{csv}-файла с метками объектов реализован скрипт на языке \smallcode{Python} \cite{cnn-age-recognize-csv}.


В результате \smallcode{csv}-файл имеет следующий вид:
\begin{lstlisting}[caption={\smallcode{csv}-файл с метками объектов}, label={lst:datasetinit},columns=fullflexible, keepspaces=true]
,	image_name,				age
0,	image_23162_25.jpg,		25
1,	image_17662_78.jpg,		78
2,	image_06600_32.jpg,		32
\end{lstlisting}


\subsection*{Нормализация размеров изображений}
Для нормализации размеров изображений реализован скрипт на языке \smallcode{Python} \cite{cnn-age-recognize-resize}


Скрипт наивно (обычное сжатие или растягивание) изменяет размер изображения до заданного значения. В данной работе все изображения нормализованы к размеру 224 на 224 пикселя.


На \hyperref[fig:resize-example]{Рисунке \ref*{fig:resize-example}} приведен пример наивной нормализации изображения до заданного знчений.
\begin{figure}[h!]
	\centering
	\includegraphics[width=1\linewidth]{../artifacts/resize_example.pdf}
	\caption{Пример сжатия изображения из датасета APPA}
	\label{fig:resize-example}
\end{figure}

\subsection*{Разделение датасета на подвыборки}
Для разделения полученнго датасета на \smallcode{test, train, valid}-подвыборки был реализован скрипт на языке \smallcode{Python} \cite{cnn-age-recognize-splitter}


Скрипт делит выборку целиком на $K$ квантилей и равномерно забирает данные для подвыборок в соотношении:
\begin{itemize}
	\item train: 70\%;
	\item validation: 20\%;
	\item test: 10\%;
\end{itemize}


В результате сбора и предварительной обработки данных был получен датасет размером 31299. Каждое изображение приведено к единому размеру 224 на 224 пикселя. Создан \smallcode{csv}-файл с метками для каждой фотографии лица.
