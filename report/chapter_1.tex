\chapter{Сбор данных}
Для решения задачи распознавания возраста человека по фотографии лица был проведён анализ доступных открытых наборов данных (датасетов), содержащих изображения лиц с аннотированным возрастом.  

\section{UTKFace}
один из наиболее распространённых наборов данных для задач оценки возраста и пола.  
Он содержит более 20\,000 изображений лиц людей в возрасте от 0 до 116 лет, снятых в различных условиях освещения, ракурсах и с разным фоном.  \cite{zhifei2017cvpr}
Каждое изображение имеет аннотацию, закодированную непосредственно в названии файла в формате:
\begin{center}
	\smallcode{[age]\_[gender]\_[race]\_[date\_time].jpg}
\end{center}
\begin{itemize}
	\item \smallcode{age} — возраст человека (целое число);
	\item \smallcode{gender} — пол (0 — мужчина, 1 — женщина);
	\item \smallcode{race} — этническая принадлежность (0–4, пять категорий);
	\item \smallcode{date\_time} — метка времени.
\end{itemize}
