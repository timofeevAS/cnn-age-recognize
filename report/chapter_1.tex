\chapter{Сбор данных}
Для решения задачи распознавания возраста человека по фотографии лица был проведён анализ доступных открытых наборов данных (датасетов), содержащих изображения лиц с аннотированным возрастом.  


\section{Публичные датасеты}
В рамках работы используются и рассматриваются следующие открытые датасеты:
\begin{itemize}
	\item UTKFace;
	\item APPA-Real Age;
\end{itemize}
\subsection*{UTKFace}
UTKFace - один из наиболее распространённых наборов данных для задач оценки возраста и пола.  
Он содержит более 20\,000 изображений лиц людей в возрасте от 0 до 116 лет, снятых в различных условиях освещения, ракурсах и с разным фоном.  \cite{zhifei2017cvpr}
Каждое изображение имеет аннотацию, закодированную непосредственно в названии файла в формате:
\begin{center}
	\smallcode{[age]\_[gender]\_[race]\_[date\_time].jpg}
\end{center}
\begin{itemize}
	\item \smallcode{age} — возраст человека (целое число);
	\item \smallcode{gender} — пол (0 — мужчина, 1 — женщина);
	\item \smallcode{race} — этническая принадлежность (0–4, пять категорий);
	\item \smallcode{date\_time} — метка времени.
\end{itemize}

Изображения имеют размер: 200 на 200 пикселей.

\subsection*{APPA-Real Age}
APPA-Real Age - содержит 7591 изображение с указанием реального и предполагаемого возраста. Общее количество предполагаемых голосов составляет около 250 000. В среднем на каждое изображение приходится около 38 голосов, что делает средний предполагаемый возраст очень стабильным (0,3 стандартной ошибки от среднего значения). \cite{agustsson2017appareal}

Каждое изображение имеет порядковый номер, а реальный возраст указан в отдельном \smallcode{csv} файле: 
\begin{itemize}
	\item \smallcode{image name} — порядковый номер изображения;
	\item \smallcode{real\_age} — реальный возраст;
	\item \smallcode{apparent\_age} — воспринимаемый возраст;
\end{itemize}


Изображения имеют различный размер.\\
В данной работе будем использовать только метки реального возраста.


\section{Предобработка данных}
Для выбранных датасетов будут выполнены следующие шаги для приведения данных к единому виду:
\begin{enumerate}
	\item Загрузка данных из открытых источников.
	\item Фильтрация лишних изображений.
	\item Сохранение файлов в единой директории с консистентным именем: \smallcode{image\_[number]\_[real\_age].jpg}
	\item TO BE DONE
\end{enumerate}